\documentclass[10pt,a4paper]{report}
\usepackage[utf8]{inputenc}
\usepackage{amsmath}
\usepackage{amsfonts}
\usepackage{amssymb}

\author{KIZITO SEBUTEMBA EZEKIEL\\
13/u/6970/eve\\
213012200\\}
\title{CHALLENGES OF LECTURE ROOMS, ATTENDANCE AND LEARNING AT COLLEGE OF COMPUTING AND INFORMATION SCIENCES (COCIS).}
\begin{document}
	\begin{center}
		\subsection*{Introduction} 
	\end{center}
\begin{flushleft}
There has been a considerable amount of literature emphasizing the role of facilitating brain development all the way up to the upright reasoning and minds. Intelligence cannot just be attributed to genes but also the environment with which we are surrounded. In higher learning institutes such as Makerere University, the learning environment plays a much significant role in the shaping of the minds of students that are yet to become upright elite citizens and face the undoubtedly steadfast growing economy.\\
In this case we are set to focus on the issues that could determine some of the challenges that come with our environment. (That’s to do with the lecture room capability, factors that affect attendance as we as the learning environment).\\
These kinds of challenges come with the intention to better our situation and be better than what we were initially. The concerns below are some of the focal point on this issue;\\
•	Does the nature/design of the lecture rooms affect student’s behavior and response towards lectures?\\

•	What are the new developments requirements students required by the design of the lecture rooms?\\

•	How can the lecture rooms be designed better to enhance or boost student attendance and learning at COCIS?\\

•	Is there a moderate way between students and the lecture rooms design, learning environment as well as the nature of learning?\\

•	How do the placements of tool use at COCIS influence student behavior?\\

•	How do students feel about the current lecture room and learning design at hand?\\

•	Is there a healthy platform to access study material from the lecturer's end by the students?\\

•	Are there unfavorable environment situations among fellow students such as their accommodation, welfare, health etc?\\

\subsection*{Background}Over the years, Makerere University has experienced tremendous number of students. The institute is characterized by a limited number of well established college structures such as those of College of Computing and Information Systems, College of Agricultural and Environmental Sciences, College of Business and Management Sciences, College of Education and External Studies College of Engineering, Design, Art and Technology,  College of Health Sciences, etc.\\
All these, however, are sub-categorized under various departments. Such departments specifically College of computing and information sciences includes Department of Computer Science, Department of Information Technology, Department of Information Systems, and Department of Networks.\\
These are further assigned with different courses that come with each particular student assigned to them. All these students have their unique IDs in the sense of registration numbers and student numbers. It’s at the point where the nature of lecture design or setting is determined.\\
Over the previous years such as 2016, students from other colleges like Education are seen moving from one lecture room to another in a bid to find a suitable lecture room and thus the possibility to attend in person. With regard to this challenge, there is a possibility to curb some of the tendencies that come along with the design of the lecture rooms put up at the different colleges and hence improve attendance thus boosting learning at the great institute.\\

\subsection*{Problem Statement}The format with which the lecture rooms are designed at the college of computing and information sciences does not entirely satisfy student learning environment. This is evident in the attendance during lectures as well as the learning environment.\\

\subsection*{Objectives}

\subsubsection*{Main}1. To devise a pro-active approach to improving number of attendances during lecture rooms.\\

2. To come up platforms that outline better lecture room designs and hence improving student-learning environment.

\subsubsection*{Specific}1. To establish a simplified means of learning environment with regards to boosting attendance in lecture rooms.\\

2. To enable an improved platform that can easily suggest on which lecture room designs best fit a particular course.

\subsubsection*{Scope}The description of this challenge mainly focuses on the ability of a simplified platform that can advise on the lecture room design per course in order to improve its attendance and thus favorable learning.

	
\subsubsection*{Methodology}As a way to determine the key causes of the challenge at hand, a means to collect data is necessary and in so doing; we focus mainly through carrying out online surveys with reference to what could be some of the causes of minimal attendance in class, effect of the class design with regards to attendance and also the learning environment.\\


\end{flushleft}
\end{document}